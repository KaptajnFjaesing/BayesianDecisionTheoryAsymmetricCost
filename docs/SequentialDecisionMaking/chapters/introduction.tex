\section{Introduction}
Inventory management is a critical component of supply chain management that involves overseeing the ordering, storage, and use of a company's inventory, including raw materials, components, and finished goods. Effective inventory control ensures that a company has adequate stock to meet demand while minimizing costs associated with holding inventory. In practice, this often requires a balance between minimizing stockouts, which disrupt service levels, and limiting overstocking, which incurs unnecessary holding costs.

The coordination of inventory and transportation decisions has gained significant attention over recent decades, driven by the need for specific practices like vendor-managed inventory (VMI), third-party logistics (3PL), and time-definite delivery (TDD) (see, e.g., \citep{CetinkayaLee2000, Alumur2012, Gurler2014}). These programs aim to optimize the balance between inventory holding costs and transportation costs. Traditional inventory models often assume immediate delivery to meet demand, yet this can be inefficient due to the fixed costs of transportation, prompting companies to adopt shipment consolidation policies that merge smaller demands into larger, less frequent shipments \citep{CetinkayaBookbinder2003, HigginsonBookbinder1994}.

Three commonly implemented shipment consolidation policies—quantity-based, time-based, and hybrid—are particularly useful in improving cost efficiency by regulating shipment size or timing. Each policy type has distinct impacts on performance metrics, such as delay penalties, average inventory, and annual costs, guiding companies toward designing more efficient inventory-transportation systems \citep{Cetinkaya2005, Wei2020, Cetinkaya2006}.

While these approaches provide valuable frameworks, they do not explicitly account for the uncertainties inherent in demand fluctuations or the associated asymmetric costs of overstocking and understocking. In settings where inventory decisions must be made sequentially, uncertainty can be more effectively managed through Bayesian decision theory, which offers a probabilistic approach to decision-making under uncertainty. This framework allows for more refined, sequential adjustments to inventory based on demand forecasting, which dynamically incorporates new data over time.

In this study, Bayesian decision theory is utilized to address a sequential inventory control problem. Specifically, an optimal decision-making framework is derived based on a discounted asymmetric cost function, considering the impact of decisions made infinitely far into the future. Our model involves a decision-maker (henceforth referred to as the Robot) who manages stock levels over time in an interactive environment shaped by a random demand process. Through this setup, a mathematically rigorous and computationally feasible solution for optimal inventory decisions that account for uncertainty and asymmetric costs is derived.