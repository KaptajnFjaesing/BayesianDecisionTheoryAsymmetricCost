
\section{Introduction}
This study explores Bayesian decision theory within a sequential decision-making framework, focusing on deriving optimal decision rules from asymmetric cost functions. Bayesian decision theory provides a probabilistic approach for making decisions under uncertainty, allowing decision-makers to account for varying costs associated with different outcomes. By incorporating asymmetric cost functions, this approach enables a more nuanced treatment of decision costs, which is crucial in fields requiring precise risk management where different actions may lead to consequences with diverse cost implications.

To illustrate these concepts, consider a specific setting in which decision-making unfolds as an interactive game between two entities: Nature, which represents the randomness in the environment, and a Robot, which operates as a decision-maker tasked with managing a finite stock of units over time. At each time iteration $t$, the Robot makes decisions based on its action space $\Omega_U$, while Nature selects outcomes from its action space $\Omega_S$. For example, if the Robot begins with an initial stock of $N_0$ units, each time step involves a certain number of units being removed by Nature ($s_t \in \Omega_S$), while the Robot has the option to reorder stock, subject to an unspecified lead time.

This stock management scenario serves as an example of sequential decision-making under uncertainty, where costs are associated with both overstocking (when $N > 0$) and understocking (when $N < 0$), which influence the Robot's decisions in each iteration. To support its decision-making, the Robot is provided with historical data $D$, containing records of past unit removals and relevant features for demand forecasting. The stock level at time $t$ is given by  
\begin{equation}
	N_t \equiv N_0 + \sum_{t'=1}^{t} (U_{t'-L} - s_{t'}),
\end{equation}
and the Robot's decisions are informed by a probabilistic forecast based on $D$ in the form  
\begin{equation}
	p(s_1, s_2, \dots | D, I),
	\label{eq:prob_forecast}
\end{equation}
where $I$ denotes any additional background information~\citep{Sivia2006}. In order to construct the asymmetric cost functions considered in this study, the swish function
\begin{equation}
	\text{swish}(N_t,\beta) = \frac{N_t}{1+e^{-N_t\beta}},
\end{equation}
with $\beta\gg 1 $, is used as a continuous imitation of Relu, meaning
\begin{equation}
	\text{swish}(\pm N_t,\beta) \simeq N_t^{(\pm)},
\end{equation}
such that
\begin{equation}
	N_t^{(\pm)}\subset C(u,s).
\end{equation}
Given the observations $D$, the Robot's objective is to formulate a set of decision rules, $\xi\equiv \{U_0(D),U_1(D),\dots\}$ with  $U_j(D)=u_j$, that minimize the expected cost associated with its decisions
\begin{equation}
	\xi^*\equiv \argmin_{\xi}(\mathbb{E}[C|D,I]),
\end{equation}
where $C$ is the cost infinitely far into the future. In this study, the cost will be assumed to be discounted, meaning
\begin{equation}
	C = \sum_{t=1}^{\infty}\gamma^{t}C_t
\end{equation}
where $C_t \equiv C(U_t(D),s_t)$ and $\gamma\in [0,1]$. The optimal decisions are then defined by
\begin{equation}
\frac{d}{dU_m}\mathbb{E}[C|D,I]\bigg|_{U_m = U_m^*} \overset{!}{=} 0\quad \forall m.
\label{eq:min_exp_cost}
\end{equation}

\begin{example}
	The cost function
	\begin{equation}
		C = \sum_{t=1}^{\infty}\gamma^{t}(k^\text{sc}_{t}N_{t}^{(+)}+k_{t}^\text{uv}N_{t}^{(-)})
	\end{equation}
	where $k^\text{sc}$ and $k^\text{uv}$ represent the "storage cost" and the "unit value" (lost value if under stocking), leads to the decision rules (see appendix \ref{app:deriva})
	\begin{equation}
		p(N_t^*\geq 0|D,I)=\frac{k_{t}^\text{uv}}{k_{t}^\text{uv}+k^\text{sc}_{t}}
		\label{eq:crit1}
	\end{equation}
	with
	\begin{equation}
		N_t^*\equiv N_0+\sum_{t'=1}^{t}(U_{t'-L}^*-s_{t'}).
	\end{equation}
\end{example}


\begin{example}
	The cost function
	\begin{equation}
		C = \sum_{t=1}^{\infty}\gamma^{t}(k^\text{sc}_{t}N_{t}^{(+)}+k_{t}^\text{uv}s_{t}N_{t}^{(-)})
	\end{equation}
	yielding the decision rule (see appendix \ref{app:deriv})
	\begin{equation}
		\frac{p(N_t\geq 0|D,I)}{\mathbb{E}[s_t|D,I]-\mathbb{E}[s_t|N_{t}\geq 0,D,I]}=\frac{k_{t}^\text{uv}}{k^\text{sc}_{t}}.
		\label{eq:crit2}
	\end{equation}
\end{example}






