\section{Optimal Policy}
\label{app:deriva}
In accordance with theorem \ref{def:policy}, the optimal policy $\pi^*$ is defined by the first and second order conditions of equation \eqref{eq:conditions}.

\subsection{First order condition}
The first order condition can be written
\begin{equation}
	\begin{split}
		\frac{d}{dU_m} \mathbb{E}[C | D, I] \Big|_{U_m = U_m^*} &= \frac{d}{dU_m} \sum_{s_1,s_2,\dots s_T}  p(s_1,s_{2},\dots s_T|D,I)\frac{dC}{dU_m}\Big|_{U_m=U_m^*}\\
		&=0
	\end{split}
	\label{eq:cond}
\end{equation}
with
\begin{equation}
		\frac{dC}{dU_m} = \sum_{t=1}^{T} \gamma_{\text{disc}}^{t-1} \left( h_t 1_{N_t> 0} + c_t (1_{N_t> 0}-1) \right)\frac{dN_t}{dU_m}
	\label{eq:deriv_1ab}
\end{equation}
and
\begin{equation}
	\begin{split}
		\frac{dN_q}{dU_m} &= \sum_{t'=1}^t\frac{dU_{t'-L}}{dU_m}\\
		& = \sum_{t'=1}^t\delta_{t'-L,m}.
	\label{eq:deriv_2ab}
	\end{split}
\end{equation}
Using equation \eqref{eq:deriv_2ab} in equation \eqref{eq:deriv_1ab} 
\begin{equation}
	\begin{split}
		\frac{dC}{dU_m} = \sum_{t=1}^{T}\gamma_{\text{disc}}^{t-1}\bigg(h_t1_{N_t> 0}+c_t(1_{N_t> 0}-1)\bigg)\sum_{t'=1}^t\delta_{t'-L,m}.
	\end{split}
\end{equation}
For some generic function $g_t$
\begin{equation}
	\begin{split}
		\sum_{t=1}^{T}g_t\sum_{t'=1}^t\delta_{t'-L,m} & = g_1\delta_{1-L,m}+g_2(\delta_{1-L,m}+\delta_{2-L,m})+\dots\\
		&=\sum_{t=L+m}^T g_t
	\end{split}
\end{equation}
meaning
\begin{equation}
	\frac{dC}{dU_m} = \sum_{t=L+m}^{T}\gamma_{\text{disc}}^{t-1}(h_t1_{N_t> 0}+c_t(1_{N_t> 0}-1)).
	\label{eq:deriv_3a}
\end{equation}
Combining equations \eqref{eq:cond} and \eqref{eq:deriv_3a}
\begin{equation}
	\sum_{s_1,s_2,\dots s_T}\sum_{t=L+m}^{T}\gamma_{\text{disc}}^{t-1}(h_t1_{N_{t}> 0}+c_t(1_{N_{t}> 0}-1))p(s_1,s_{2},\dots s_T|D,I)\Big|_{U_m=U_m^*} = 0
\end{equation}
The sums can be evaluated viz
\begin{equation}
	\begin{split}
		\sum_{s_1,s_2,\dots s_T}1_{N_{t}> 0}p(s_1,s_{2},\dots s_T|D,I) &= p(N_t> 0|D,I),\\
		\sum_{s_1,s_2,\dots s_T}p(s_1,s_{2},\dots s_T|D,I)&=1.\\
	\end{split}
\end{equation}
Let
\begin{equation}
	\psi_t\equiv (h_t+c_t)p(N_t> 0|D,I)-c_t,
	\label{eq:2}
\end{equation} 
then
\begin{equation}
	\begin{split}
		\frac{d}{dU_m}\mathbb{E}[C|D,I]\Big|_{U_m=U_m^*}& = \sum_{t=L+m}^{T}\gamma_{\text{disc}}^{t-1}\psi_t\\
		&= 0
	\end{split}
\end{equation}
A recursion relation can be derived viz
\begin{equation}
	\begin{split}
		\frac{d}{dU_0}\mathbb{E}[C|D,I]& = \sum_{t=L}^{T}\gamma_{\text{disc}}^{t-1}\psi_t\\
		& =\gamma_{\text{disc}}^{L-1}\psi_L+\sum_{t=L+1}^{T}\gamma_{\text{disc}}^{t-1}\psi_t\\
		& =\gamma_{\text{disc}}^{L-1}\psi_L+\frac{d}{dU_1}\mathbb{E}[C|D,I]\\
		& =\gamma_{\text{disc}}^{L-1}\psi_L+\gamma_{\text{disc}}^{L}\psi_{L+1}+\frac{d}{dU_2}\mathbb{E}[C|D,I]\\
		&=\dots\\
	\end{split} 
\end{equation}
meaning
\begin{equation}
		\frac{d}{dU_m}\mathbb{E}[C|D,I] =\gamma_{\text{disc}}^{L+m-1}\psi_{L+m}+\frac{d}{dU_{m+1}}\mathbb{E}[C|D,I]. 
\end{equation}
For the optimal policy, all first order derivative vanish, meaning
\begin{equation}
	\forall t\geq L, \quad \gamma_{\text{disc}}^{t-1}\psi_t|_{\pi=\pi^*}= 0 \Rightarrow \psi_t|_{\pi=\pi^*}=0.
	\label{eq:1}
\end{equation}
Combining equations \eqref{eq:1} and \eqref{eq:2} yields
\begin{equation}
	p(N_t^*> 0|D,I)=\frac{c_t}{c_t+h_t},
	\label{decision_rules}
\end{equation}
for $t\geq L$ and $\upsilon_t^*\geq 0$, where
\begin{equation}
	N_t^*\equiv N_0+\upsilon_t^*-\zeta_t
\end{equation}
denote the units on stock given optimal decisions.

\subsection{Second order condition}
For $N_t = 0$, the indicator function $1_{N_t > 0}$ transitions from $1$ to $0$, leading to a discontinuity in the derivative of the cost function (equation \ref{eq:deriv_3a}). For this reason, the second order derivative is not well defined at this point. However, since $\frac{dC}{dU_m}>0$ for $N_t > 0$ and $\frac{dC}{dU_m}<0$ for $N_t < 0$ it can be inferred that the cost function achieves a local minimum in the neighborhood of $N_t=0$.
