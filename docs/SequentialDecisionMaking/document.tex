\documentclass[a4paper,12pt]{article}
\usepackage{amsmath}  % For equations
\usepackage{amsfonts} % For mathematical symbols
\usepackage{graphicx} % For figures
\usepackage{hyperref} % For links
\usepackage{appendix} % For appendices
\usepackage [round,authoryear] {natbib}

\DeclareMathOperator*{\argmax}{argmax}
\DeclareMathOperator*{\argmin}{argmin}

\usepackage{amsthm}
\usepackage{xcolor}

\theoremstyle{definition}
\newtheorem{exinn}{Example}

\newenvironment{example}
{\clubpenalty=10000
	\begin{exinn}%
		\mbox{}%
		{\color{black}\leaders\hrule height .8ex depth \dimexpr-.8ex+0.8pt\relax\hfill}%
		\mbox{}\linebreak\ignorespaces}
	{\par\kern2ex\hrule\end{exinn}}


\title{Sequential Decision-Making for Stock Management}
\author{Jonas Petersen}
\date{\today}

\begin{document}
	\maketitle
	
	\begin{abstract}
		This study explores Bayesian decision theory within a sequential decision-making framework, focusing on deriving optimal decision rules from an asymmetric cost functions representing stock management.
	\end{abstract}
	
	% Main Content
	\noindent
	Bayesian decision theory provides a probabilistic approach for making decisions under uncertainty, allowing decision-makers to account for varying costs associated with different outcomes. By incorporating asymmetric cost functions, this approach enables a more nuanced treatment of decision costs, which is crucial in fields requiring precise risk management where different actions may lead to consequences with diverse cost implications.

	This study explores Bayesian decision theory within a sequential decision-making framework, focusing on deriving optimal decision rules from asymmetric cost functions. Bayesian decision theory provides a probabilistic approach for making decisions under uncertainty, allowing decision-makers to account for varying costs associated with different outcomes. By incorporating asymmetric cost functions, this approach enables a more nuanced treatment of decision costs, which is crucial in fields requiring precise risk management where different actions may lead to consequences with diverse cost implications.
	
	To illustrate these concepts, consider a specific setting in which decision-making unfolds as an interactive game between two entities: Nature, which represents the randomness in the environment, and a Robot, which operates as a decision-maker tasked with managing a finite stock of units over time. At each time iteration $t$, the Robot makes decisions based on its action space $\Omega_U$, while Nature selects outcomes from its action space $\Omega_S$. For example, if the Robot begins with an initial stock of $N_0$ units, each time step involves a certain number of units being removed by Nature ($s_t \in \Omega_S$), while the Robot has the option to reorder stock, subject to an unspecified lead time.
	
	This stock management scenario serves as an example of sequential decision-making under uncertainty, where costs are associated with both overstocking (when $N > 0$) and under stocking (when $N < 0$), which influence the Robot's decisions in each iteration. To support its decision-making, the Robot is provided with historical data $D$, containing records of past unit removals and relevant features for demand forecasting. The stock level at time $t$ is given by  
	\begin{equation}
		N_t \equiv N_0 + \sum_{t'=1}^{t} (U_{t'-L} - s_{t'}),
	\end{equation}
	and the Robot's decisions are informed by a probabilistic forecast based on $D$ in the form  
	\begin{equation}
		p(s_1, s_2, \dots | D, I),
		\label{eq:prob_forecast}
	\end{equation}
	where $I$ denotes any additional background information~\citep{Sivia2006}. In order to construct the asymmetric cost functions considered in this study, the relu function
	\begin{equation}
		\begin{split}
			\text{relu}(\pm N_t) &= \max(\pm N_t,0)\\
			&\equiv N_t^{(\pm)}
		\end{split},
	\end{equation}
	is used, such that
	\begin{equation}
		N_t^{(\pm)}\subset C(u,s).
	\end{equation}
	Given data, $D$, the Robot's objective is to formulate a set of decision rules, $\xi\equiv \{U_0(D),U_1(D),\dots\}$ with  $U_j(D)=u_j$, that minimize the expected cost associated with its decisions
	\begin{equation}
		\xi^*\equiv \argmin_{\xi}(\mathbb{E}[C|D,I]),
	\end{equation}
	where $C$ is the cost infinitely far into the future. In this study, the cost will be assumed to be discounted, meaning
	\begin{equation}
		C = \sum_{t=1}^{\infty}\gamma^{t}(k^\text{sc}_{t}N_{t}^{(+)}+k_{t}^\text{uv}N_{t}^{(-)})
	\end{equation}
	where $\gamma\in [0,1]$ and $k^\text{sc}$ and $k^\text{uv}$ represent the "storage cost" and the "unit value" (lost value if under stocking). The optimal decisions are then defined by
	\begin{equation}
		\frac{d}{dU_m}\mathbb{E}[C|D,I]\bigg|_{U_m = U_m^*} \overset{!}{=} 0\quad \forall m
		\label{eq:min_exp_cost}
	\end{equation}
	which leads to the decision rules (see appendix \ref{app:deriva})
	\begin{equation}
		p(N_t^*\geq 0|D,I)=\frac{k_{t}^\text{uv}}{k_{t}^\text{uv}+k^\text{sc}_{t}}
		\label{eq:crit1}
	\end{equation}
	with
	\begin{equation}
		N_t^*\equiv N_0+\sum_{t'=1}^{t}(U_{t'-L}^*-s_{t'}).
	\end{equation}
	From equation \eqref{eq:crit1} the optimal decisions can be determined viz
	\begin{equation}
		\sum_{t'=1}^{t}U_{t'-L}^* = Q_{100\frac{k_{t}^\text{uv}}{k^\text{sc}_{t}+k_{t}^\text{uv}}}\bigg(\sum_{t'=1}^{t}s_{t'}-N_0\bigg),
	\end{equation}
	where $Q_q(X)$ is the $q$-quantile for the random variable $X$.


	
	\newpage
	\begin{appendices}
		\section{Minimization of Expected Cost}
\label{app:deriva}
Define the cost function viz
\begin{equation}
	C = \sum_{t=1}^{\infty}\gamma^{t}(k^\text{sc}_{t}N_{t}^{(+)}+k_{t}^\text{uv}N_{t}^{(-)}),
\end{equation}
where $k^\text{sc}$ and $k^\text{uv}$ represent the "storage cost" and the "unit value". To determine the decision rules, $\xi^*$, that minimize the expected cost $\mathbb{E}[C|D,I]$, the derivative of the cost function with respect to $\xi$ is needed
\begin{equation}
		\frac{dC}{dU_m} = \sum_{t=1}^{\infty}\gamma^{t}\bigg(k^\text{sc}_{t}\frac{dN_{t}^{(+)}}{dU_m}+k_{t}^\text{uv}\frac{dN_{t}^{(-)}}{dU_m}\bigg)
	\label{eq:deriv_1ab}
\end{equation}
where
\begin{equation}
	\begin{split}
		\frac{dN_{t}^{(+)}}{dU_m}& =\sum_q 	\frac{dN_{t}^{(+)}}{dN_q}\frac{dN_q}{dU_m}\\
		& \simeq\sum_q 	1_{N_{t}\geq 0}\delta_{t,q}\frac{dN_q}{dU_m}\\
		& = 1_{N_{t}\geq 0}\sum_{t'=1}^t\frac{dU_{t'-L}}{dU_m}\\
		& = 1_{N_{t}\geq 0}\sum_{t'=1}^t\delta_{t'-L,m},\\
		\frac{dN_{t}^{(-)}}{dU_m} &\simeq (1_{N_t\geq 0}-1)\sum_{t'=1}^t\delta_{t'-L,m},
		\label{eq:deriv_2ab}
	\end{split}
\end{equation}
and it has been used that
\begin{equation}
	\begin{split}
		\frac{dN_t^{(+)}}{dN_t} & =\frac{\beta N_te^{-\beta N_t}}{(1+e^{-\beta N_t})^2}+\frac{1}{1+e^{-\beta N_t}}\\
		& \simeq \frac{1}{1+e^{-\beta N_t}}\\
		&= \sigma(\beta N_t)\\
		&\simeq 1_{N_{t}\geq 0},\\
		\frac{dN_t^{(-)}}{dN_t} & = \frac{\beta N_te^{\beta N_t}}{(1+e^{\beta N_t})^2}-\frac{1}{1+e^{\beta N_t}}\\
		& \simeq -\frac{1}{1+e^{\beta N_t}}\\
		&= \sigma(\beta N_t)-1\\
		& \simeq  1_{N_{t}\geq 0}-1.
	\end{split}
\end{equation}
Using equation \eqref{eq:deriv_2ab} in equation \eqref{eq:deriv_1ab} 
\begin{equation}
	\begin{split}
		\frac{dC}{dU_m} \simeq \sum_{t=1}^{\infty}\gamma^{t}\bigg(k^\text{sc}_{t}1_{N_{t}\geq 0}+k_{t}^\text{uv}(1_{N_{t}\geq 0}-1)\bigg)\sum_{t'=1}^t\delta_{t'-L,m}.
	\end{split}
\end{equation}
For some generic function $g_t$
\begin{equation}
	\begin{split}
		\sum_{t=1}^{\infty}g_t\sum_{t'=1}^t\delta_{t'-L,m} & = g_1\delta_{1-L,m}+g_2(\delta_{1-L,m}+\delta_{2-L,m})+\dots\\
		&=\sum_{t=L+m}^\infty g_t
	\end{split}
\end{equation}
meaning
\begin{equation}
	\frac{dC}{dU_m} \simeq \sum_{t=L+m}^{\infty}\gamma^{t}(k^\text{sc}_{t}1_{N_{t}\geq 0}+k_{t}^\text{uv}(1_{N_{t}\geq 0}-1)).
	\label{eq:deriv_3a}
\end{equation}
Combining equations \eqref{eq:min_exp_cost} and \eqref{eq:deriv_3a}
\begin{equation}
	\sum_{s_1,s_2,\dots}\sum_{t=L+m}^{\infty}\gamma^{t}(k^\text{sc}_{t}1_{N_{t}\geq 0}+k_{t}^\text{uv}(1_{N_{t}\geq 0}-1))p(s_1,s_{2},\dots|D,I)\overset{!}{=} 0\quad \forall m
\end{equation}
The sums can be evaluated viz
\begin{equation}
	\begin{split}
		\sum_{s_1,s_2,\dots}1_{N_{t}\geq 0}p(s_1,s_{2},\dots|D,I) &= p\bigg(\sum_{t'=1}^{t}s_{t'}\leq N_0+\sum_{t'=1}^{t}U_{t'-L}|D,I\bigg)\\
		&= p(N_t\geq 0|D,I),\\
		\sum_{s_1,s_2,\dots}p(s_1,s_{2},\dots|D,I)&=1.\\
	\end{split}
\end{equation}
Let
\begin{equation}
	\psi_t\equiv (k^\text{sc}_{t}+k_{t}^\text{uv})p(N_t\geq 0|D,I)-k_{t}^\text{uv},
\end{equation} 
then
\begin{equation}
	\begin{split}
		\frac{d}{dU_m}\mathbb{E}[C|D,I]& = \sum_{t=L+m}^{\infty}\gamma^{t}\psi_t\\
		&\overset{!}{=} 0\quad \forall m
	\end{split}
\end{equation}
A recursion relation can be derived viz
\begin{equation}
	\begin{split}
		\frac{d}{dU_0}\mathbb{E}[C|D,I] & = \sum_{t=L}^{\infty}\gamma^{t}\psi_t\\
		& =\gamma^{L}\psi_L+\sum_{t=L+1}^{\infty}\gamma^{t}\psi_t\\
		& =\gamma^{L}\psi_L+\frac{d}{dU_1}\mathbb{E}[C|D,I]\\
		& =\gamma^{L}\psi_L+\gamma^{L+1}\psi_{L+1}+\frac{d}{dU_2}\mathbb{E}[C|D,I]\\
		&=\dots\\
		&\overset{!}{=} 0
	\end{split} 
\end{equation}
Since all derivatives are required to be simultaneously zero,
\begin{equation}
	\gamma^{j}\psi_j\overset{!}{=} 0\quad \forall j \Rightarrow \psi_j=0
\end{equation}
meaning
\begin{equation}
	p(N_t^*\geq 0|D,I)=\frac{k_{t}^\text{uv}}{k^\text{sc}_{t}+k_{t}^\text{uv}},
\end{equation}
where
\begin{equation}
	N_t^*\equiv N_0+\sum_{t'=1}^{t}(U_{t'-L}^*-s_{t'})
\end{equation}
denote the units on stock given optimal decisions.

	\end{appendices}
	
	
	
	\bibliographystyle{plainnat}
	\bibliography{ref}
	
\end{document}